\documentclass[a4paper]{article}
\usepackage[T1]{fontenc}
\usepackage[utf8]{inputenc}

% Align text to left
\raggedright

% Headings
\usepackage{titlesec}

\titleformat{\section}[display]
  {\large\sffamily\bfseries}{\thesection}{1cm}{}

% Font settings
\usepackage[tracking]{microtype}
\usepackage[sc,osf]{mathpazo}
\linespread{1.025}
\usepackage[euler-digits,small]{eulervm}
\AtBeginDocument{\renewcommand{\hbar}{\hslash}}

% Landscape orientation
\usepackage[margin=2cm, landscape]{geometry}

% Multi-column document
\usepackage{multicol}

% No page numbering
\pagenumbering{gobble}

% Less vertical spacing
\raggedcolumns

% Coding environment
\usepackage{listings}
\lstset{basicstyle=\small\ttfamily}
\lstMakeShortInline|

% Math packages (AmS)
\usepackage{amsmath}
\usepackage{amssymb}

% For examples
\usepackage{mathtools}
\DeclarePairedDelimiter{\abs}{\lvert}{\rvert}
\DeclarePairedDelimiter{\norm}{\lVert}{\rVert}
\usepackage{bm}

\begin{document}
% Title
{\centering \textsc{\huge \AmS-\LaTeX\ Quick Reference} \par}
\vspace{1mm}
\rule{\textwidth}{0.2mm}
\begin{multicols*}{3}
\section*{Packages}
The main package to load is |amsmath|.
More symbols are included in |amssymb|.

\section*{Typeset}
For text style (inline) math, use: |$| \dots\ |$| \\
For display style math, which breaks the paragraph: \\
|\begin{equation}| \dots\ |\end{equation}| (numbered) or |\[| \dots\ |\]| (non-numbered).

\section*{Greek letters}
\begin{tabular}{clclcl}
  $\alpha$ & |\alpha| & $\beta$ & |\beta| & $\gamma$ & |\gamma| \\
  $\delta$ & |\delta| & $\epsilon$ & |\epsilon| & $\varepsilon$ & |\varepsilon| \\
  $\zeta$ & |\zeta| & $\eta$ & |\eta| & $\theta$ & |\theta| \\
  $\vartheta$ & |\vartheta| & $\iota$ & |\iota| & $\kappa$ & |\kappa| \\
  $\lambda$ & |\lambda| & $\mu$ & |\mu| & $\nu$ & |\nu| \\
  $\xi$ & |\xi| & $\pi$ & |\pi| & $\varpi$ & |\varpi| \\
  $\rho$ & |\rho| & $\varrho$ & |\varrho| & $\sigma$ & |\sigma| \\
  $\tau$ & |\tau| & $\upsilon$ & |\upsilon| & $\phi$ & |\phi| \\
  $\varphi$ & |\varphi| & $\chi$ & |\chi| & $\psi$ & |\psi| \\
  $\omega$ & |\omega| \\
  $\Gamma$ & |\Gamma| & $\Delta$ & |\Delta| & $\Theta$ & |\Theta| \\
  $\Lambda$ & |\Lambda| & $\Xi$ & |\Xi| & $\Pi$ & |\Pi| \\
  $\Sigma$ & |\Sigma| & $\Upsilon$ & |\Upsilon| & $\Phi$ & |\Phi| \\
  $\Psi$ & |\Psi| & $\Omega$ & |\Omega|
\end{tabular}\\
\vspace{2mm}
To ensure a consistent style throughout the document: \\
\vspace{1mm}
|\renewcommand{\epsilon}{\varepsilon}| \\
|\renewcommand{\theta}{\vartheta}| \\
|\renewcommand{\rho}{\varrho}| \\
|\renewcommand{\phi}{\varphi}|

\section*{Mathematical font}
$\mathcal{A} \, \mathcal{B} \, \mathcal{C} \, \mathcal{D} \, \mathcal{E} \, \mathcal{F} \, \mathcal{G} \, \mathcal{H} \, \mathcal{I} \, \mathcal{J} \, \mathcal{K} \, \mathcal{L} \, \mathcal{M} \, \mathcal{N} \, \mathcal{O} \, \mathcal{P} \, \mathcal{Q} \, \mathcal{R} \, \mathcal{S} \, \mathcal{T} \, \mathcal{U} \, \mathcal{V} \, \mathcal{W} \, \mathcal{X} \, \mathcal{Y} \, \mathcal{Z}$ \\
\vspace{1mm}
|\mathcal{| \emph{letter} |}|

\section*{Superscript \& Subscript}
\begin{tabular}{clccl}
  $x^y$ & |x^y| & $\quad$ & $x^{a+b}$ & |x^{a+b}| \\
  $x_y$ & |x_y| & $\quad$ & $x_{a+b}$ & |x_{a+b}|
\end{tabular}

\section*{Root}
\begin{tabular}{rcl}
  Square root & $\sqrt{x}$ & |\sqrt{x}| \\
  N-th root & $\sqrt[N]{x}$ & |\sqrt[N]{x}|
\end{tabular}

\section*{Dots}
\begin{tabular}{rcl}
  Multiplication dot & $\cdot$ & |\cdot| \\
  Three centered dots & $\cdots$ & |\cdots| \\
  Three baseline dots & $\ldots$ & |\ldots| \\
  Three diagonal dots & $\ddots$ & |\ddots| \\
  Three vertical dots & $\vdots$ & |\vdots|
\end{tabular}

\section*{Spaces}
\begin{tabular}{rl}
  Negative space & |\!| \\
  Thinnest space & |\,| \\
  Thin space & |\:| \\
  Medium space & |\;| \\
  1em space & |\quad| \\
  2em space & |\qquad|
\end{tabular}

\section*{Braces}
|\overbrace{| \dots\ |}^{| \emph{text over brace} |}| \\
|\underbrace{| \dots\ |}_{| \emph{text under brace} |}|

\section*{Accents}
\begin{tabular}{clclcl}
  $\hat{a}$ & |\hat{a}| & $\bar{a}$ & |\bar{a}| & $\mathring{a}$ & |\mathring{a}| \\
  $\check{a}$ & |\check{a}| & $\dot{a}$ & |\dot{a}| & $\vec{a}$ & |\vec{a}| \\
  $\tilde{a}$ & |\tilde{a}| & $\ddot{a}$ & |\ddot{a}| & $\widehat{AAA}$ & |\widehat{AAA}|
\end{tabular}

\section*{Operators}
\begin{tabular}{lllll}
  |\sin| & |\cos| & |\arcsin| & |\arccos| & |\sinh| \\
  |\cosh| & |\tan| & |\arctan| & |\log| & |\ln|\\
  |\max| & |\min| & |\sup| & |\inf| & |\tanh|\\
  |\cot| & |\sec| & |\csc| & |\det|
\end{tabular}\\
\vspace{2mm}
To define a custom operator:\\
|\DeclareMathOperator{\argmax}{argmax}|

\section*{Modulo}
\begin{tabular}{rl}
  $a \bmod b$ & |a \bmod b| \\
  $a \equiv b \pmod{m}$ & |a \equiv b \pmod{m}|
\end{tabular}

\section*{Fractions}
|\frac{| \dots\ |}{| \dots\ |}|

\section*{Symbol stacking}
|\overset{| \dots\ |}{| \dots\ |}| $\qquad$ |\underset{| \dots\ |}{| \dots\ |}| \\
\vspace{1mm}
First argument is the main symbol, second argument is the symbol to put over or under the main symbol.

\section*{Big operators}
\bgroup
  \def\arraystretch{2.3}
  \begin{tabular}{clcl}
    $\displaystyle \int_{a}^{b}$ & |\int_{a}^{b}| & $\displaystyle \sum_{k=0}^{n}$ & |\sum_{k=0}^{n}| \\
    $\displaystyle \prod_{k=0}^{n}$ & |\prod_{k=0}^{n}| & $\displaystyle \lim_{x \to 0}$ & |\lim_{x \to 0}|
  \end{tabular}
\egroup

\vspace{3mm}
For multiple integrals: $\iint$ |\iint| $\,\, \iiint$ |\iiint| etc.
Closed path integral: $\oint$ |\oint|

\section*{Delimiter size}
Change the delimiter size by adding one of these modifiers immediately before the delimiter itself: \\
|\big \Big \bigg \Bigg|\\
\vspace{2mm}
Let \LaTeX\ determine the correct size using |\left| and |\right| immediately before the opening and closing delimiters, respectively.

\section*{Absolute value \& Norm}
\begin{tabular}{cl}
  $\lvert x \rvert$ & |\lvert x \rvert| \\
  $\lVert x \rVert$ & |\lVert x \rVert|
\end{tabular}

\vspace{3mm}
The same can be achieved by defining: \\
\vspace{1mm}
|\usepackage{mathtools}| \\
|\DeclarePairedDelimiter{\abs}{\lvert}{\rvert}| \\
|\DeclarePairedDelimiter{\norm}{\lVert}{\rVert}| \\

\vspace{3mm}
Use starred variants |\abs*| and |\norm*| to produce the correct delimiter height for any kind of equation.
\vspace{3mm}

\bgroup
\def\arraystretch{1.2}
\begin{tabular}{clcl}
  $\abs{x}$ & |\abs{x}| & $\abs*{\frac{a}{b}}$ & |\abs*{\frac{a}{b}}| \\
  $\norm{x}$ & |\norm{x}|& $\norm*{\frac{a}{b}}$ & |\norm*{\frac{a}{b}}|
\end{tabular}
\egroup

\section*{Arrows}
\begin{tabular}{clclcl}
  $\uparrow$ & |\uparrow| & $\downarrow$ & |\downarrow| & $\updownarrow$ & |\updownarrow| \\
  $\Uparrow$ & |\Uparrow| & $\Downarrow$ & |\Downarrow| & $\Updownarrow$ & |\Updownarrow|
\end{tabular}

\begin{tabular}{clcl}
  $\leftarrow$ & |\leftarrow| or |\gets| & $\rightarrow$ & |\rightarrow| or |\to| \\
  $\leftrightarrow$ & |\leftrightarrow| & $\Leftarrow$ & |\Leftarrow| \\
  $\Rightarrow$ & |\Rightarrow| & $\Leftrightarrow$ & |\Leftrightarrow| \\
  $\mapsto$ & |\mapsto|
\end{tabular}

{\centering
\begin{tabular}{cl}
  $\longleftarrow$ & |\longleftarrow| \\
  $\longrightarrow$ & |\longrightarrow| \\
  $\longleftrightarrow$ & |\longleftrightarrow| \\
  $\Longleftarrow$ & |\Longleftarrow| \\
  $\Longrightarrow$ & |\Longrightarrow| \\
  $\Longleftrightarrow$ & |\Longleftrightarrow| \\
  $\longmapsto$ & |\longmapsto|
\end{tabular}
\par}

\section*{Binary relations}
\begin{tabular}{clclclcl}
  $\ne$ & |\ne| & $\le$ & |\le| & $\ge$ & |\ge| \\
  $\equiv$ & |\equiv| & $\ll$ & |\ll| & $\gg$ & |\gg| \\
  $\doteq$ & |\doteq| & $\sim$ & |\sim| & $\simeq$ & |\simeq| \\
  $\subset$ & |\subset| & $\supset$ & |\supset| & $\approx$ & |\approx| \\
  $\subseteq$ & |\subseteq| & $\supseteq$ & |\supseteq| & $\cong$ & |\cong| \\
  $\in$ & |\in| & $\ni$ & |\ni| & $\propto$ & |\propto| \\
  $\mid$ & |\mid| & $\parallel$ & |\parallel| & $\perp$ & |\perp|
\end{tabular}

\vspace{3mm}
It's possible to negate these symbols by prefixing them with |\not| (for example: $\not\equiv$ |\not\equiv|)

\section*{Binary operators}
\begin{tabular}{clclcl}
  $\pm$ & |\pm| & $\mp$ & |\mp| & $\cdot$ & |\cdot| \\
  $\div$ & |\div| & $\times$ & |\times| & $\setminus$ & |\setminus| \\
  $\star$ & |\star| & $\cup$ & |\cup| & $\cap$ & |\cap| \\
  $\ast$ & |\ast| & $\circ$ & |\circ| & $\bullet$ & |\bullet| \\
  $\oplus$ & |\oplus| & $\ominus$ & |\ominus| & $\odot$ & |\odot| \\
  $\oslash$ & |\oslash| & $\otimes$ & |\otimes| & $\smallsetminus$ & |\smallsetminus|
\end{tabular}

\section*{Logic symbols}
\begin{tabular}{clclcl}
  $\lor$ & |\lor| & $\land$ & |\land| & $\neg$ & |\neg| \\
  $\exists$ & |\exists| & $\nexists$ & |\nexists| & $\forall$ & |\forall| \\
  $\implies$ & |\implies| & $\iff$ & |\iff| & $\models$ & |\models|
\end{tabular}

\section*{Other symbols}
\begin{tabular}{rcl}
  Infinity & $\infty$ & |\infty| \\
  Partial derivative & $\partial$ & |\partial| \\
  Empty set & $\emptyset$ & |\emptyset| \\
  Nabla & $\nabla$ & |\nabla| \\
  Angle brackets & $\langle x \rangle$ & |\langle x \rangle|
\end{tabular}

\section*{Multi line equations}
Use the |multline| environment: \\
|\begin{multline}| \dots\ |\end{multline}| \\

\vspace{2mm}
To align equations, use the |align| environment. Specify the alignment position with |&| and separate equations with |\\|: \\

\vspace{1mm}
|\begin{align}| \\
\dots\ |&=| \dots |\\| \\
\dots\ |&=| \dots \\
|\end{align}|

\section*{Vectors}
\begin{tabular}{cl}
  $\vec{x}$ & |\vec{x}| \\
  $\bm{x}$ & |\bm{x}| (needs |bm| package)
\end{tabular}

\vspace{2mm}
Best practice to easily switch between types: \\
|\usepackage{bm}| \\
|\renewcommand{\vec}{\bm}|

\section*{Arrays}
Use the |array| environment. Use |\\| to separate rows, and |&| to separate elements of each row. To produce large delimiters around the array, use |\left| and |\right| followed by the desired delimiter. \\

\begin{center}
\begin{minipage}[c]{3cm}
  $\left(
  \begin{array}{lcr}
    a & b & c \\
    d & e & f \\
    g & h & i
  \end{array}
  \right)$
\end{minipage}
\hspace{0.1cm}
\begin{minipage}[c]{2.5cm}
  |\left(| \\
  |\begin{array}{lcr}| \\
  |  a & b & c \\| \\
  |  d & e & f \\| \\
  |  g & h & i| \\
  |\end{array}| \\
  |\right)|
\end{minipage}
\end{center}

Each letter in the argument of the array represents a column: \\
\vspace{2mm}
\begin{tabular}{cl}
  |l| & left aligned text \\
  |c| & centered text \\
  |r| & right aligned text
\end{tabular}

\section*{Cases}
Use the |cases| environment. Use |\\| to separate different cases, and |&| for correct alignment. \\

\vspace{3mm}
\begin{minipage}[c]{3cm}
  $\quad
  \begin{cases}
    x & \text{if } x > 0 \\
    0 & \text{if } x \le 0
  \end{cases}$
\end{minipage}
\hspace{0.1cm}
\begin{minipage}[c]{2cm}
  |\begin{cases}| \\
  |  x & \text{if } x > 0 \\| \\
  |  0 & \text{if } x \le 0| \\
  |\end{cases}|
\end{minipage}

\columnbreak
\section*{Matrices}
Use one of the following environments: \\

\vspace{2mm}
$\qquad$
\begin{tabular}{rl}
  |matrix| & No delimiter \\
  |pmatrix| & $($ delimiter \\
  |bmatrix| & $[$ delimiter \\
  |Bmatrix| & $\{$ delimiter \\
  |vmatrix| & $\lvert$ delimiter \\
  |Vmatrix| & $\lVert$ delimiter
\end{tabular}

\vspace{3mm}
Use |\\| to separate different rows, and |&| to separate elements of each row.
\begin{center}
\begin{minipage}[c]{3cm}
  $\begin{bmatrix}
    1 & 2 & 3 \\
    4 & 5 & 6 \\
  \end{bmatrix}$
\end{minipage}
\hspace{0.2cm}
\begin{minipage}[c]{3cm}
  |\begin{bmatrix}| \\
  |  1 & 2 & 3 \\| \\
  |  4 & 5 & 6| \\
  |\end{bmatrix}|
\end{minipage}
\end{center}

\vspace{1mm}
To produce a small matrix, useful for inline math, use the |smallmatrix| environment: $\left[\begin{smallmatrix} a & b \\ c & d \end{smallmatrix}\right]$.

\end{multicols*}
\end{document}
